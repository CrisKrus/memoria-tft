\section{Tecnologías y herramientas utilizadas}
\subsection{Tecnologías}
\begin{itemize}
    \item\textbf{Ionic:} \textit{framework} de desarrollo para aplicaciones móviles, este es
    de código abierto. Está basado en varias tecnologías entre ellas HTML, Typescrip, SCCS. La
    ventaja de este \textit{framework} para el desarrollo multiplataforma es el tener que
    desarrollar solamente una versión de la aplicación y luego esta se puede exportar a IOS,
    Android o Windows Phone. Dada la importancia de esta tecnología, se expondrán las razones
    por las que se ha decidido usar próximamemnte en esta misma sección.
    \item\textbf{Ionic View App:} aplicación móvil con la que poder visualizar el proyecto que
    estás desarrollando, para poder hacer uso de ella tienes que subir a la web de ionic tu
    proyecto, una vez ahí te dan un código único identificador que introduces en tu teléfono
    tras esto te cargará la aplicación como si la tuvieses instalada y podrás probarla de manera
    nativa.
    \item\textbf{Angular:} \textit{framework} de \textit{JavaScript} de código abierto mantenido
    por Google usado principalmente para la creación de \textit{SPA}\cite{SPA} de manera más fácil
    a las existentes en la actualidad, además este te permite una creación de pruebas de forma más
    sencilla.
    \item\textbf{Cordova:} entorno de desarrollo para aplicaciones móviles el cual está basado en
    HTML, JavaScript y CSS, facilita el empaquetado de todo esto dependiendo del dispositivo y
    hace que no tengas que conocer la \textit{API}\cite{api} de cada plataforma por separado y
    hacer un tratamiento individual para cada una de ellas.
    \item\textbf{Typescript:} lenguaje de programación basado en JavaScript el cual mejora las
    desventajas o problemas que pudiera tener JavaScrip añadiendo un tipado estático y clases,
    entre otras mejoras. Esto hace que su uso sea más cómodo y al estar basado en JavaScript si
    hiciera falta escribir partes de código en JavaScrip puro se puede hacer sin problemas.
    %    \item\textbf{JavaScript:}
    \item\textbf{HTML5:} lenguaje básico de elaboración de páginas web. Es un estándar con el que
    estructurar las distintas páginas/ventanas de nuestra aplicación. Este se considera el lenguaje
    más importante para el crecimiento de la \textit{World Wide Web} (WWW), y ha sido adoptado por
    todos los navegadores actuales para la visualización de páginas webs.
    \item\textbf{SCSS:} \textit{Sassy CSS} al igual que con \textit{TypeScrip} es una ampliación del
    CSS clásico con mejoras tales como la creación de variables de manera más cómoda e intuitiva.
    Al ser una ampliación de CSS se puede escribir CSS clásico en él y lo interpreta correctamente.
    \item\textbf{Firebase:} plataforma con múltiples funcionalidades para la creación de aplicaciones
    móviles, en ella podemos encontrar sistemas de autenticación, base de datos, almacenamiento de
    datos, etc. Se ha decidido usar esta por su fácil uso y gran cantidad de documentación de calidad
    de la que dispone.
    \item\textbf{Git:} sistema de control de versiones con el que gestionar la evolución de la
    aplicación guardando los estados de todos los ficheros por los que va pasando durante el
    desarrollo, de esta manera si hubiese algún problema se puede deshacer los cambios hasta ese
    punto.
\end{itemize}

\subsubsection{Ionic}
\textit{Ionic} es la base de esta propuesta. Es por ello que a continuación se expondrán algunas
de las razones por la que se ha decidido usar este \textit{framework}.

\begin{itemize}
    \item El código se escribe una vez y se usa en todas
    las plataformas, con ionic solo se tiene que escribir
    una vez la lógica que va a seguir y esta puede ser
    compilada luego en la plataforma que se requiera.
    \item Los componentes por defecto que trae para el
    desarrollo de la interfaz dan un buen acabado, sin
    tener que modificar la mayoría de ellos. Lo que permite
    ahorrar tiempo en este apartado.
    \item Se puede comprobar el comportamiento de manera rápida
    y fácil ya que permite ejecutar la aplicación en un
    explorador como si se tratase de un dispositivo móvil.
    Además incluye un apartado donde se puede ver las tres
    versiones de compilación y comprobar el comportamiento
    en cada una de ellas por separado sin tener un dispositivo
    donde instalar la aplicación.
    \item Gran cantidad de documentación oficial y no oficial
    así como tutoriales escritos como en vídeo, de buena calidad.
    \item Lo apoya una gran comunidad y de calidad, lo que hace
    junto al punto anterior que la búsqueda de una solución a
    problemas que puedan surgir sea rápida y eficaz.
    \item Trabaja con tecnologías modernas, con lo que
    su arquitectura es limpia y robusta.
    \item Al estar basado en lenguajes básicos y estandarizados
    hace que su uso resulte en ciertos puntos intuitivo.
    \item Lleva varios años en el mercado funcionando y tiene
    cientos de aplicaciones hechas con él, por lo que no es un
    producto tan nuevo como para que pueda tener grandes fallos.
\end{itemize}

\subsection{Herramientas}
\begin{itemize}
    %    \item\textbf{Firefox:} explorador web en el que previsualizar la aplicación durante el desarrollo
    \item\textbf{WebStorm:} IDE para trabajar con tecnolgías web, desarrollado por la empresa
    \textit{JetBrains}\cite{jetbrains}. Cuenta con editores para HTML, JavaScript, PHP,
    Typescript, SCSS entre otros además de autocompletado, funciones de refactorización
    automática y depurador.
    \item\textbf{StarUML:} programa de creación de diagramas, soporta distintos tipos de estos
    diagramas de clases, diagramas de casos de uso, etc.
    \item\textbf{Github:} portal web el cual soporta el sistema de control de versiones de git,
    donde poder subir tu proyecto gestionado con git y ver de manera visual todo lo que ello conlleva
    además el proyecto puede ser público, si se quiere, pudiendo cualquier persona que lo vea
    sugerir mejoras a este o posibles cambios.
\end{itemize}

\subsection{Servidores}
\begin{itemize}
    \item\textbf{Firebase, base de datos en tiempo real:} para el guardado de datos se ha usado
    una base de datos no relacional que provee \textit{firebase} por las ventajas que esta proporciona
    frente a una relacional. La principal por la que esta se tuvo en cuenta ha sido que permite
    datos sean variables, puedan cambiar a lo largo del tiempo sin tener que para la base de datos
    debido a que aún el cliente no tiene claro como quiere guardar los ejercicios, además estas
    bases de datos permiten mayor accesibilidad al consumir menos recursos para su mantenimiento.
\end{itemize}
