\section{Tecnologías y herramientas utilizadas}
\subsection{Tecnologías}
\begin{itemize}
    \item\textbf{Ionic:} \textit{framework} de desarrollo para aplicaciones móviles, este es
    de código abierto. Está basado en varias tecnologías entre ellas HTML, Typescrip, SCCS. La
    ventaja de este \textit{framework} para el desarrollo multiplataforma es el tener que
    desarrollar solamente una versión de la aplicación y luego esta se puede exportar a IOS,
    Android o Windows Phone.
    \item\textbf{Ionic View App:} aplicación móvil con la que poder visualizar el proyecto que
    estás desarrollando, para poder hacer uso de ella tienes que subir a la web de ionic tu
    proyecto, una vez ahí te dan un código único identificador que introduces en tu teléfono
    tras esto te cargará la aplicación como si la tuvieses instalada y podrás probarla de manera
    nativa.
    \item\textbf{Angular:} \textit{framework} de \textit{JavaScript} de código abierto mantenido
    por Google usado principalmente para la creación de \textit{SPA}\cite{SPA} de manera más fácil
    a las existentes en la actualidad, además este te permite una creación de pruebas de forma más
    sencilla.
    \item\textbf{Cordova:} entorno de desarrollo para aplicaciones móviles el cual está basado en
    HTML, JavaScript y CSS, facilita el empaquetado de todo esto dependiendo del dispositivo y
    hace que no tengas que conocer la \textit{API}\cite{api} de cada plataforma por separado y
    hacer un tratamiento individual para cada una de ellas.
    \item\textbf{Typescript:} lenguaje de programación basado en JavaScript el cual mejora las
    desventajas o problemas que pudiera tener JavaScrip añadiendo un tipado estático y clases,
    entre otras mejoras. Esto hace que su uso sea más cómodo y al estar basado en JavaScript si
    hiciera falta escribir partes de código en JavaScrip puro se puede hacer sin problemas.
    %    \item\textbf{JavaScript:}
    \item\textbf{HTML5:} lenguaje básico de elaboración de páginas web. Es un estándar con el que
    estructurar las distintas páginas/ventanas de nuestra aplicación. Este se considera el lenguaje
    más importante para el crecimiento de la \textit{World Wide Web} (WWW), y ha sido adoptado por
    todos los navegadores actuales para la visualización de páginas webs.
    \item\textbf{SCSS:} \textit{Sassy CSS} al igual que con \textit{TypeScrip} es una ampliación del
    CSS clásico con mejoras tales como la creación de variables de manera más cómoda e intuitiva.
    Al ser una ampliación de CSS se puede escribir CSS clásico en él y lo interpreta correctamente.
    \item\textbf{Firebase:} plataforma con múltiples funcionalidades para la creación de aplicaciones
    móviles, en ella podemos encontrar sistemas de autenticación, base de datos, almacenamiento de
    datos, etc. Se ha decidido usar esta por su fácil uso y gran cantidad de documentación de calidad
    de la que dispone.
    \item\textbf{Git:} sistema de control de versiones con el que gestionar la evolución de la
    aplicación guardando los estados de todos los ficheros por los que va pasando durante el
    desarrollo, de esta manera si hubiese algún problema se puede deshacer los cambios hasta ese
    punto.
\end{itemize}

\subsection{Herramientas}
\begin{itemize}
    \item\textbf{Firefox:}
    \item\textbf{WebStorm:} IDE para trabajar con tecnolgías web, desarrollado por la empresa
    \textit{JetBrains}\cite{jetbrains}. Cuenta con editores para HTML, JavaScript, PHP,
    Typescript, SCSS entre otros además de autocompletado, funciones de refactorización
    automática y depurador.
    \item\textbf{StarUML:}
    \item\textbf{Github:}
\end{itemize}

\subsection{Servidores}
\begin{itemize}
    \item\textbf{Firebase:}
\end{itemize}
