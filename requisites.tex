\section{Requisitos}
Para representar los requisitos de la aplicación se ha usado UML\cite{uml} por varios
motivos que se expondrán a continuación. El primero de ellos por estar familiarizados
con este, al haber sido usado durante el grado en múltiples ocasiones en asignaturas
distintas. Por otra parte este lenguaje de modelado es un estándar aprobado por la ISO
lo que hace que cualquier diagrama creado pueda ser interpretado de igual forma por
diferentes personas que conozcan el estándar.

\medskip
A continuación se muestran los diagramas UML con los que se han representado los
requisitos de los distintos actores de la aplicación. Solo se ha querido representar los
requisitos que se quieren implementar en esta fase para no cargar con más información de
la necesaria hasta el momento. En el apartado mejoras futuras se nombrarán el resto de ellos.

\medskip
\begin{figure}
    \includegraphics[width=\linewidth]{./images/patient.jpeg}
    \caption{Diagrama casos de uso paciente.}
    \label{Diagrama casos de uso paciente.}
\end{figure}

\medskip
\begin{figure}
    \includegraphics[width=\linewidth]{./images/doctor.jpeg}
    \caption{Diagrama casos de uso doctor.}
    \label{Diagrama casos de uso doctor.}
\end{figure}

\medskip
\begin{figure}
    \includegraphics[width=\linewidth]{./images/user.jpeg}
    \caption{Diagrama casos de uso usuarios.}
    \label{Diagrama casos de uso usuarios.}
\end{figure}

\medskip
En la siguiente tabla se muestran los casos de uso anteriores con una breve descripción.

\begin{tabular}{|llp{12cm}|}
    \hline
    ID & ACTOR    & DESCRIPCIÓN \\ \hline
    1  & Usuario  & Puede registrarse en la aplicación \\ \hline
    2  & Usuario  & Puede iniciar sesión en la aplicaicón \\ \hline
    3  & Usuario  & Puede cerrar la sesión \\ \hline
    4  & Usuario  & Puede crear un chat con otra persona \\ \hline
    5  & Usuario  & Puede ver el chat ya creado o que otro usuario ha creado con él \\ \hline
    6  & Usuario  & Puede enviar mensajes por el chat que ha creado u otro usuario
    ha creado co él \\ \hline
    7  & Usuario  & Puede enviar un vídeo en formato mp4 por el chat \\ \hline
    8  & Usuario  & Puede ver el vídeo que le han enviado \\ \hline
    9  & Doctor   & Buscar pacientes en el sistema para asignarle ejercicios o
    consultar su información \\ \hline
    10 & Doctor   & Puede ver el perfil de un paciente que ha buscado \\ \hline
    11 & Doctor   & Puede asignar un ejercicio a un paciente para que este le quede
    reflejadoq que tiene que hacerlo \\ \hline
    12 & Doctor   & En el momento de asignar un ejercicio puede añadir observaciones
    si lo considera el doctor \\ \hline
    13 & Doctor   & Puede consultar la evolución del paciente mirando los ejercicios
    que ha realizado y con que frecuencia \\ \hline
    14 & Paciente & Puede ver el ejercicio que se le ha asignado para consultar las
    observaciones o el ejercicio en sí \\ \hline
    15 & Paciente & Puede marcar como completado un ejercicio que le ha sido asignado \\ \hline
\end{tabular}


% especificacion de casos de uso no la pongo, no hay casos de uso complejos como para explicar

% arquitectura del sistema
%    front
%    back

