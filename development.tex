\section{Desarrollo}
\subsection{Ionic}
Se expondrá a continuación las razones por las que se
ha decidido usar ionic como framework de desarrollo para
este proyecto:

\begin{itemize}
    \item El código se escribe una vez y se usa en todas
    las plataformas, con ionic solo se tiene que escribir
    una vez la lógica que va a seguir y esta puede ser
    compilada luego en la plataforma que se requiera.
    \item Los componentes por defecto que trae para el
    desarrollo de la interfaz dan un buen acabado, sin
    tener que modificar la mayoría de ellos. Lo que permite
    ahorrar tiempo en este apartado.
    \item Se puede comprobar el comportamiento de manera rápida
    y fácil ya que permite ejecutar la aplicación en un
    explorador como si se tratase de un dispositivo móvil.
    Además incluye un apartado donde se puede ver las tres
    versiones de compilación y comprobar el comportamiento
    en cada una de ellas por separado sin tener un dispositivo
    donde instalar la aplicación.
    \item Gran cantidad de documentación oficial y no oficial
    así como tutoriales escritos como en vídeo, de buena calidad.
    \item Lo apoya una gran comunidad y de calidad, lo que hace
    junto al punto anterior que la búsqueda de una solución a
    problemas que puedan surgir sea rápida y eficaz.
    \item Trabaja con tecnologías modernas, con lo que
    su arquitectura es limpia y robusta.
    \item Al estar basado en lenguajes básicos y estandarizados
    hace que su uso resulte en ciertos puntos intuitivo.
    \item Lleva varios años en el mercado funcionando y tiene
    cientos de aplicaciones hechas con él, por lo que no es un
    producto tan nuevo como para que pueda tener grandes fallos.
\end{itemize}

\subsection{Estructura del proyecto}

Uno de los puntos buenos que tiene este \textit{framework} es
poseer una estructura de directorios bien conocida y definida,
por ello será fácil navegar por ella incluso para nuevos
desarrolladores que se incorporasen al proyecto.

\medskip
Los principales directorios son los siguientes:

\begin{itemize}
    \item /src/app/: ficheros de configuración de la aplicación.
    \item /src/assets/: ficheros multimedia y recursos estáticos.
    \item /src/enviroments/: ficheros de configuración para los
    distintos entornos (desarrollo o producción)
    \item /src/pages/: directorio donde van ubicados los directorios
    de cada pandalla o componente de la aplicación.
    \item /src/pages/page1/: ficheros referentes a una pantalla en
    concreto en el van el HTML, TS, SCSS (si fuese necesario)
    \item /src/providers/: ficheros de acceso a los datos.
    \item /src/theme/: fichero de estilo genérico en el cual
    se basa la aplicación.
\end{itemize}

% hablar de los distintos tipos de ficheros que hay en el poryecto
% meter trozos de codigo explicando como va el sistema
% y hacer tiempo para hacer un par de paginas mas
% contar la importacia de los buenos nombres