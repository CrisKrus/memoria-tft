\section{Análisis}
\subsection{Actores}
Se han identificado dos actores principales los cuales se expondrán a continuación.

\subsubsection{Doctor}
Personal del centro médico el cual tendrá pacientes a su cargo, sus funciones serán
las siguientes:
\begin{itemize}
    \item Buscar pacientes entre los registrados en el sistema.
    \item Asignar ejercicios a un paciente.
    \item Añadir observaciones a los ejercicios que se van a asignar.
    \item Ver perfil de un paciente.
    \item Ver evolución de un paciente.
\end{itemize}

Toda funcionalidad tienen como objetivo que el doctor pueda indicarle a sus
pacientes los ejercicios que este tiene que realizar y si en su caso concreto tendrá
que tener en cuenta alguna variación del original dejarlo indicado en las observaciones.
Además tendrá que poder ver el perfil del paciente para poder consultar sus datos, en caso
de que le fueran necesarios.

\subsubsection{Paciente}
Otro actor que se ha identificado es el paciente con la siguiente funcionalidad:
\begin{itemize}
    \item Ver ejercicio.
    \item Marcar ejercicio como hecho.
\end{itemize}

Con esta funcionalidad se busca que el paciente pueda llevar un control de los ejercicios
que ha realizado y esto pueda ser consultado por el doctor en cualquier momento.

\medskip
Además estos actores principales tendrán funcionalidades \textbf{comunes:}

\begin{itemize}
    \item Registro en la aplicación.
    \item Inicio de sesión.
    \item Ver perfil.
    \item Editar perfil.
    \item Iniciar chat.
    \item Enviar y visualizar vídeos por el chat.
\end{itemize}

Estas funciones son necesarias que estén disponibles en ambos actores. Debido a que el chat
es tanto doctor con paciente, como a la inversa. Además, los usuarios tienen que poder
darse de alta en el sistema.