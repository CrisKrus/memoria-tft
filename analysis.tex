\section{Análisis}
\subsection{Actores}
Se han identificado dos actores principales los cuales se expondrán a continuación.

\subsubsection{Doctor}
Personal del centro médico el cual tendrá pacientes a su cargo, sus funciones serán
las siguientes:
\begin{itemize}
    \item Buscar pacientes entre los registrados en el sistema.
    \item Asignar ejercicios a un paciente.
    \item Añadir observaciones a los ejercicios que se van a asignar.
    \item Ver perfil de un paciente.
    \item Ver evolución de un paciente.
\end{itemize}

Todas estas funcionalidades tienen como objetivo que el doctor pueda indicarle a sus
pacientes los ejercicios que este tiene que realizar y si en su caso concreto tendrá
que tener en cuenta alguna variación del original dejarlo indicado en las observaciones.
Además tendrá que poder ver el perfil del paciente para poder consultar sus datos, en caso
de que le fueran necesarios.

\subsubsection{Paciente}
Otro actor que se ha identificado es el paciente con las siguiente funcionalidad:
\begin{itemize}
    \item Ver ejercicio.
    \item Marcar ejercicio como hecho.
\end{itemize}

Con esta funcionalidad se busca que el paciente pueda llevar un control de los ejercicios
que ha realizado y esto pueda ser consultado por el doctor en cualquier momento.

\medskip
Además estos actores principales tendrán funcionalidades comunes:

\begin{itemize}
    \item Registro en la aplicación.
    \item Inicio de sesión.
    \item Ver perfil.
    \item Editar perfil.
    \item Iniciar chat.
    \item Enviar y visualizar vídeos por el chat.
\end{itemize}

Estas funciones son necesarias que estén disponibles en ambos actores dado que el chat
es doctor con paciente y viceversa, además tienen que poder darse de alta en el sistema.

\section{Requisitos}
Para representar los requisitos de la aplicación se ha usado UML\cite{uml} por varios
motivos que se expondrán a continuación. El primero de ellos por estar familiarizados
con este, al haber sido usado durante el grado en múltiples ocasiones en asignaturas
distintas. Por otra parte este lenguaje de modelado es un estándar aprobado por la ISO
lo que hace que cualquier diagrama creado pueda ser interpretado de igual forma por
diferentes personas que conozcan el estándar.

\medskip
A continuación se muestran los diagramas UML con los que se han representado los
requisitos de los distintos actores de la aplicación. Solo se ha querido representar los
requisitos que se quieren implementar en esta fase para no cargar con más información de
la necesaria hasta el momento. En el apartado mejoras futuras se nombrarán el resto de ellos.

\medskip
\begin{figure}
    \includegraphics[width=\linewidth]{./images/patient.jpeg}
    \caption{Diagrama casos de uso paciente.}
    \label{Diagrama casos de uso paciente.}
\end{figure}

\medskip
\begin{figure}
    \includegraphics[width=\linewidth]{./images/doctor.jpeg}
    \caption{Diagrama casos de uso doctor.}
    \label{Diagrama casos de uso doctor.}
\end{figure}

\medskip
\begin{figure}
    \includegraphics[width=\linewidth]{./images/user.jpeg}
    \caption{Diagrama casos de uso usuarios.}
    \label{Diagrama casos de uso usuarios.}
\end{figure}

\medskip
En la siguiente tabla se muestran los casos de uso anteriores con una breve descripción.

\begin{tabular}{|| lll ||}
    ID & ACTOR    & DESCRIPCIÓN \\
    1  & Usuario  & Puede registrarse en la aplicación \\
    2  & Usuario  & Puede iniciar sesión en la aplicaicón \\
    3  & Usuario  & Puede cerrar la sesión \\
    4  & Usuario  & Puede crear un chat con otra persona \\
    5  & Usuario  & Puede ver el chat ya creado o que otro usuario ha creado con él \\
    6  & Usuario  & Puede enviar mensajes por el chat que ha creado u otro usuario ha creado co él \\
    7  & Usuario  & Puede enviar un vídeo en formato mp4 por el chat \\
    8  & Usuario  & Puede ver el vídeo que le han enviado \\
    9  & Doctor   & Buscar pacientes en el sistema para asignarle ejercicios o consultar su información \\
    10 & Doctor   & Puede ver el perfil de un paciente que ha buscado \\
    11 & Doctor   & Puede asignar un ejercicio a un paciente para que este le quede reflejadoq que tiene que hacerlo \\
    12 & Doctor   & En el momento de asignar un ejercicio puede añadir observaciones si lo considera el doctor \\
    13 & Doctor   & Puede consultar la evolución del paciente mirando los ejercicios que ha realizado y con que frecuencia \\
    14 & Paciente & Puede ver el ejercicio que se le ha asignado para consultar las observaciones o el ejercicio en sí \\
    15 & Paciente & Puede marcar como completado un ejercicio que le ha sido asignado \\
\end{tabular}

% especificacion de casos de uso no la pongo, no hay casos de uso complejos como para explicar

% arquitectura del sistema
%    front
%    back

% base de datos no relacional, diseño de la misma

% Diseño arquitectonico el que viene con ionic por defecto, mirar cual es
