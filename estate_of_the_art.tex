\section{Introducción}
%  ESTADO  ACTUAL  Y  OBJETIVOS INICIALES: situación actual del tema relacionado 
%  con el TFT, motivación y objetivos que se pretenden cubrir con este trabajo 
%  (deberán recogerse obligatoriamente los objetivos inicialmente planteados en el TFT-01)
El día 7 de abril de 2005 en la 58ª asamblea mundial de la salud la Organización Mundial
de la Salud \cite{OMS} trató la temática de la cibersalud (conocida también como e-Salud
o e-Health) y su importancia. Indicando que esta "consiste en el apoyo que
la utilización costoeficaz y segura de las tecnologías de la información y las comunicaciones
ofrece a la salud y a los ámbitos relacionados con ella, con inclusión de los servicios de
atención de salud, la vigilancia y la documentación sanitaria, así como la educación, los
conocimientos y las investigaciones en materia de salud" \cite{58-asamblea}.

Dados los avances de las nuevas tecnologías y que estas están presentes en facetas de
nuestra vida diaria como: educación, comunicación, producción industrial. ¿Porqué no
en la medicina? aplicaciones como
BilliCam \cite{BilliCam} con la que se puede sacar una foto al bebé con una tarjeta de
calibración en el vientre. Con esto la app distingue la luz y el tono de la piel del recién
nacido, y con ello busca el diagnóstico de la ictericia neonatal. Como esta hay otras
aplicaciones las cuales buscan solucionar problemas muy específicos.

\medskip
Patient Portal by ConstantMD \cite{patient-portal} esta app permite  almacenar citas
en varios de centros y actividades. Además, permite tener ordenadas las citas y las
reservas que se hayan realizado.

\medskip
CoreFusion Pilates & Physio \cite{coreFusion} tiene como objetivo objetivo mostrar a
los usuarios el estado de las clases de Pilates para así poder saber cuales están
libres y cuáles pueden ser reservadas.

\medskip
Otras como \textit{Connect Physiotherapy} \cite{Connect-Physiotherapy} permiten al usuario
indicar la zona que quiere rehabilitar y ver ejercicios relacionados con ella para fortalecer
dicha zona. Esta tiene un objetivo similar a la propuesta, con la diferencia de que
\textit{Connect Physiotherapy} está centrada en pacientes de media y avanzada edad.

\medskip
Estas aplicaciones se acercan a la propuesta cada en su ámbito. \textit{BilliCam} ayudando
a los niños prematuros a solucionar un problema del diagnóstico de la icteria neonatal.
\textit{Patient Protal} y \textit{CoreFusio} solucionan el problema de gestión de citas, un
objetivo que a largo plazo se podría plantear en la propuesta. Y finalmente
\textit{Connect Physiotherapy} es la que más se acerca, ya que tiene un objetivo similar
pero esta solamente contiene ejercicios enfocados para personas adultas. Es por ello que
se considera que es necesaria esta aplicación, ya que trata de solucionar un problema que
ninguna de ellas ha tenido en cuenta, los ejercicios para niños prematuros. Se podría pensar
que esto se pude solucionar con una mejora de \textit{Connect Physiotherapy} añadiendo nuevos
ejercicios para niños prematuros. Pero esta se
queda corta en otros apartados tales como el poder mandar vídeos de la realización del
ejercicio para que el doctor pueda revisar la evolución del paciente.

\bigskip
En la actualidad el índice de niños prematuros ha aumentado considerablemente. Estos niños presentan
unas características únicas y es por ello que necesitan de unos cuidados especiales. Uno de los
ámbitos que requieren es la fisioterapia, que irá enfocada a estimular, facilitar, etc. las
diferentes etapas del desarrollo motor, favoreciendo su desarrollo y previniendo posibles
complicaciones a lo largo de su crecimiento, sobretodo en su primer año de vida.

Para conseguir este fin, el Servicio Canario de Salud dispone de un servicio de fisioterapia
para toda la isla que atiende y realiza un seguimiento de los casos en los que se requiere de
una asistencia profesional.

Esta asistencia conlleva la realización de una serie de terapias, consistentes en
la ejecución de una serie de ejercicios con el bebé que varían en función de su edad y su
evolución. Además, se encomienda a los padres continuar con la terapia en casa, realizando
los ejercicios que el profesional aconseja. Esto supone un reto de gran dificultad para los
padres dada la cantidad de ejercicios y posturas diferentes que deben recordar. Esto hace que
los padres busquen comunicarse con los profesionales para que estos comprueben si están
realizando correctamente el ejercicio concreto mediante el envío de vídeos por plataformas
externas. Además, dada la gran demanda del servicio, en ocasiones se ven desbordados y la
frecuencia con la que pueden citar a los bebés es muy inferior a la deseada, no pudiendo
realizar un seguimiento de forma óptima. Por ello esta propuesta en conjunto
con con una alumna de fisioterapia será de gran ayuda tanto
a pacientes como a doctores. Esta alumna se encargará de preparar el
contenido visual de la propuesta, como pueden ser diagramas y vídeos
explicativos de los ejercicios.

\bigskip
Dado que en esta aplicación es de vital importancia el material visual del que dispondrán
los usuarios. Se ha decidido hacer este proyecto en colaboración con Kenya Fiesenig Álvaro y
María del Mar Batista Guerra, alumna y profesora del grado en fisioterapia. Ellas serán de gran
ayuda para que este proyecto pueda salir adelante.

\bigskip
\subsection{Objetivos}
El alcance de la \textbf{\myTitle} tiene los siguientes objetivos:
\begin{enumerate}
    \item Poner a disposición de los padres material gráfico y multimedia de apoyo que permita desarrollar los ejercicios indicados por el fisioterapeuta.
    \item Grabar vídeos e instantáneas a los padres para que puedan registrar la evolución del bebé de forma que dicho material pueda ser supervisado por el fisioterapeuta.
    \item La comunicación Fisioterapeuta – Padres.
    \item Realizar una valoración de la evolución del bebé.
    \item Evaluar la utilidad de los materiales por parte de los padres.
\end{enumerate}

Dado el tiempo del que se dispone para la realización de esta propuesta, se ha decidido abarcar
dos de los objetivos básicos, los cuales son el punto número 1 y el número 2. Con esto los padres
podrán realizar los ejercicios indicados con mayor seguridad teniendo a su disposición material
de apoyo y, por su parte el médico podrá ver la evolución del bebé para prevenir cualquier otro
problema que pueda surgir.
