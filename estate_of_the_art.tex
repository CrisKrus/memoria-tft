  \section{Estado del arte}
%  ESTADO  ACTUAL  Y  OBJETIVOS INICIALES: situación actual del tema relacionado 
%  con el TFT, motivación y objetivos que se pretenden cubrir con este trabajo 
%  (deberán recogerse obligatoriamente los objetivos inicialmente planteados en el TFT-01)
El pasado 7 de abril de 2005 en la 58ª asamblea mundial de la salud la Organización Mundial de la Salud \cite{OMS} trató el la temática de la cibersalud (conocida también como e-Salud o e-Health) y su importancia en la actualidad. Indicando que esta "consiste en el apoyo que la utilización costoeficaz y segura de las tecnologías de la información y las comunicaciones ofrece a la salud y a los ámbitos relacionados con ella, con inclusión de los servicios de atención de salud, la vigilancia y la documentación sanitarias, así como la educación, los conocimientos y las investigaciones en materia de salud"\cite{58-asamblea}.

Dados los avances de las nuevas tecnologías y que estas están presentes en facetas  de nuestra vida diaria como: educación, comunicación, producción industrial. ¿Porqué no en la medicina? No decimos que estas no existan, porque las hay, aplicaciones como BilliCam \cite{BilliCam} con la que puedes sacar una foto a tu bebé con una tarjeta de calibración en el vientre. Con esto la app distingue la luz y el tono de la piel del recién nacido, y con ello busca el diagnóstico de la ictericia neonatal. Como esta hay otras aplicaciones las cuales buscan solucionar problemas muy específicos.

Este no es el ámbito que nosotros nos incumbe, lo que buscamos lograr es ayudar, tanto a los médicos como a pacientes el seguimiento y realización, respectivamente, de ejercicios para el apoyo del bebé en sus primeros meses. Algo que supondría un cambio a mejor en el sistema actual de atención fisioterapéutica facilitando a ambas partes.

\bigskip
Actualmente el índice de niños prematuros ha aumentado considerablemente. Estos niños presentan unas características únicas y es por ello que necesitan de unos cuidados especiales. Uno de los ámbitos que requieren es la fisioterapia, que irá enfocada a estimular, facilitar, etc. las diferentes etapas del desarrollo motor, favoreciendo su su desarrollo y previniendo posibles complicaciones a lo largo de su crecimiento, sobretodo en su primer año de vida.

Para conseguir este fin, el Servicio Canario de Salud dispone de un servicio de fisioterapia para toda la isla que atiende y realiza un seguimiento de los casos en los que se requiere de una asistencia profesional.

En general, esta asistencia conlleva la realización de una serie de terapias, consistentes en la ejecución de una serie de ejercicios con el bebé que varían en función de su edad y su evolución. Además, se encomienda a los padres continuar con la terapia en casa, realizando los ejercicios que el profesional aconseja. Esto supone un reto de gran dificultad para los padres dada la cantidad de ejercicios y posturas diferentes que deben recordar. Esto hace que los padres busquen comunicarse con los profesionales para que estos comprueben si están realizando correctamente el ejercicio concreto mediante el envío de vídeos por plataformas externas. Además, dada la gran demanda del servicio, en ocasiones se ven desbordados y la frecuencia con la que pueden citar a los bebés es muy inferior a la deseada, no pudiendo realizar un seguimiento de forma óptima. Por último, dado que el servicio se ofrece para la atención de niños prematuros de todos los municipios de la isla, el desarrollo de una herramienta informática de apoyo puede ser de gran utilidad para evitar, en ocasiones, el desplazamiento de las familias.

\bigskip
\subsection{Objetivos}
Nuestra \textbf{\myTitle} tiene como objetivos básicos:
 \begin{enumerate}
\item Poner a disposición de los padres material gráfico y multimedia de apoyo que permita desarrollar los ejercicios indicados por el fisioterapeuta. 
\item Grabar vídeos e instantáneas a los padres para que puedan registrar la evolución del bebé de forma que dicho material pueda ser supervisado por el fisioterapeuta.
\item La comunicación Fisioterapeuta – Padres.
\item Realizar una valoración de la evolución del bebé.
\item Evaluar la utilidad de los materiales por parte de los padres.
\end{enumerate}

Dado el tiempo del que hemos dispuesto para la realización de la misma, hemos decidido abarcar dos de los objetivos más relevantes para esta primera fase de la aplicación, los cuales son el punto número 1 y el número 2. Con los que los padres ya podrán realizar los ejercicios indicados con mayor seguridad teniendo a su disposición material de apoyo y, por su parte el médico podrá ver la evolución del bebé para prevenir cualquier otro problema que pueda surgir.
