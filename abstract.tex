\begin{abstract}
Nowadays the rate of premature children has increased
considerably. These children have some  unique
characteristics and that is why they need special treatments.
One of the areas that require is physiotherapy, which will go
focused on stimulating, facilitating, the different stages of the
motor development, favoring its development and preventing
possible complications throughout its growth, above all
in his first year of life.

To achieve this goal, the Servicio Canario de Salud has
of a physiotherapy service for the whole island that serves
and keeps track of the cases in which it is required
of professional assistance.

In general, this assistance entails the realization of a
series of therapies, consisting of the execution of a series
of exercises with the baby that vary according to their age and
its evolution. In addition, parents are entrusted with continuing
with home therapy, doing the exercises that the
professional advises. This is a challenge of great difficulty
for parents given the amount of exercises and postures
different that you must remember. Also, given the high demand
of the service, sometimes they are overwhelmed and the frequency
with which they can cite babies is much less than
desired, not being able to track optimally.
Finally, given that the service is offered for the attention of
neonates of all the municipalities of the island, the development of
a computer support tool can be very useful
to avoid, on occasion, the displacement of families.
\end{abstract}

\begin{abstract}
Actualmente el índice de niños prematuros ha aumentado
considerablemente. Estos niños presentan unas características
únicas y es por ello que necesitan de unos cuidados especiales.
Uno de los ámbitos que requieren es la fisioterapia, que irá
enfocada a estimular, facilitar, las diferentes etapas del
desarrollo motor, favoreciendo su su desarrollo y previniendo
posibles complicaciones a lo largo de su crecimiento, sobretodo
en su primer año de vida.

Para conseguir este fin, el Servicio Canario de Salud dispone
de un servicio de fisioterapia para toda la isla que atiende
y realiza un seguimiento de los casos en los que se requiere
de una asistencia profesional.

En general, esta asistencia conlleva la realización de una
serie de terapias, consistentes en la ejecución de una serie
de ejercicios con el bebé que varían en función de su edad y
su evolución. Además, se encomienda a los padres continuar
con la terapia en casa, realizando los ejercicios que el
profesional aconseja. Esto supone un reto de gran dificultad
para los padres dada la cantidad de ejercicios y posturas
diferentes que deben recordar. Además, dada la gran demanda
del servicio, en ocasiones se ven desbordados y la frecuencia
con la que pueden citar a los bebés es muy inferior a la
deseada, no pudiendo realizar un seguimiento de forma óptima.
Por último, dado que el servicio se ofrece para la atención de
neonatos de todos los municipios de la isla, el desarrollo de
una herramienta informática de apoyo puede ser de gran utilidad
para evitar, en ocasiones, el desplazamiento de las familias.
\end{abstract}