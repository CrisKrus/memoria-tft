\section{Normativa y legislación}

\subsection{Licencia Software}
Una licencia software se define como "\textit{un contrato entre el
licenciante (autor/titular de los derechos de explotación/distribución) y el
licenciatario (usuario consumidor, profesional o empresa) del programa
informático, para utilizarlo cumpliendo una serie de términos y condiciones
establecidas dentro de sus cláusulas,​ es decir, es un conjunto de permisos
que un desarrollador le puede otorgar a un usuario en los que tiene la
posibilidad de distribuir, usar o modificar el producto bajo una licencia
determinada. Además se suelen definir los plazos de duración, el territorio
donde se aplica la licencia (ya que la licencia se soporta en las leyes
particulares de cada país o región), entre otros.}" \cite{licencia}


\subsection{Licencia pública general de GNU}
Esta licencia \cite{GNU} existe para garantizar la libertad de compartir y modificar el
software que se encuentra bajo ella. De esta forma busca que que el software
sea libre para todos los usuarios. Otras acciones distintas de su copia no
están cubiertas por esta licencia, tales como ejecutar el programa. Dentro del
software que se usa en esta propuesta, se encuentra bajo esta licencia: Git.

\subsection{Licencia comercial}
La licencia comercial también conocida como software propietario, es aquella que
se vende bajo unas normas detalladas y condiciones de uso específicas. Estas
licencias tienen como objetivo omitir el acceso de forma libre a su código fuente,
el cual se encuentra a dispocición de su desarrollador. Además no se permite que
sea modificado. Bajo esta licencia se encuentra StarUML y Firebase.

En el caso de firebase es gratuito su uso hasta cierta cantidad de peticiones
mensuales, una vez se sobrepasa este límite tiene planes de uso según las
necesidades del proyecto en cuestión.

\subsection{MIT}
Esta licencia es bastante flexible, por lo que tiene pocas limitaciones a la hora
de su reutilización esto lo hace una licencia con una muy buena compatibilidad. La
licencia MIT permite la reutilización dentro de software propietario y es compatible
con GNU, por ejemplo. Con esta licencia nos encontramos en el proyecto a Ionic y Angular.

\subsection{Licencia para educación JetBrains}
La licencia de JetBrains para educación es de formato de suscripción mensual o anual,
dependiendo de como el usuario quiera realizar el pago. Esta es gratuita para los
miembros de la comunidad universitaria, sin embargo tiene algunas restricciones como
puede ser el uso para propósito comercial, el uso para ingeniería inversa, modificar
o descubrir el código fuente de los productos. Con esta licencia se encuentra el
entorno de desarrollo WebStorm.

\subsection{Reglamento General de Protección de Datos}
El RGPD es el nuevo reglamento que entró en vigor en mayo de 2016, de obligada
aplicación en todas las empresas de la Unión Eurpea desde el 25 de mayo de 2018.
Esta otorga mayor control y seguridad a los usuarios sobre su información personal
ampliando sus derechos a decidir cómo quieren que sus datos sean tratados y cómo
quieren recibir la informaciójn de las empresas.

\medskip
Entre los puntos de este reglamento se encuentra el derecho al olvido. Este derecho
lo pueden solicitar los usuarios y una vez se aplica los datos personales deben de ser
suprimidos cuando ya no sean necesarios para el fin con el que fueron reunidos. Otro
de los puntos es el derecho a la portabilidad. Este implica que los datos de los que
dispone el sistema tienen que ser extraibles de este por el usuario para que permita
el trasnlado a otro si así lo quisiera. El Reglamento pide que el consentimiento,
con carácter general, sea libre, informado, específico e inequívoco. Las empresas
deberán revisar la forma en la que obtienen y guardan el consentimiento.
Se exige que el consentimiento tenga que ser
“manifiesto” en determinados casos, como puede ser para autorizar el
tratamiento de datos sensibles. Por tanto, el consentimiento tiene que ser
verificable y quienes recopilen datos personales deben poder probar que el
afectado les concedió su consentimiento. Es por todo esto que para que la aplicación
cumpla con la ley se debe asegurar los datos personales tales como: nombre, apellidos,
teléfono, etc. Estos datos se encuentran en la base de datos protegidos por una contraseña
a la cual sólo tendrá acceso el administrador de la misma.