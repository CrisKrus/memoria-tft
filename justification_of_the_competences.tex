\section{Justificación de las competencias}
%  JUSTIFICAIÓN DE LAS COMPETENCIAS ESPECÍFICAS CUYBIERTAS: indicar, sólo para
%  las competencias específicas relacionadas de forma más directa con el trabajo
%  desarrollado, cómo se han cubierto con este TFT.
% COMPETENCIAS: CII08 – CII012 – CII016 – IS01 – IS02 – IS03 – IS04

\subsection{Comunes a la Ingeniería informática}

\subsubsection{CII08}
\textit{"Capacidad para analizar, diseñar, construir y mantener aplicaciones de forma
robusta, segura y eficiente, eligiendo el paradigma y los lenguajes de programación más
adecuados."}

\medskip
A lo largo del proyecto se han abarcado todas las fases de un proyecto: análisis, diseño
y desarrollo del mismo. Además, se ha tenido que hacer un estudio de las tecnologías
disponibles hasta el momento para el desarrollo y decidir cual sería la mejor.

\subsubsection{CII012}
\textit{"Conocimiento y aplicación de las características, funcionalidades y estructura de las
bases de datos, que permitan su adecuado uso, y el diseño y el análisis
e implementación de aplicaciones basadas en ellos."}

\medskip
Se ha decidido la estructura de la base de datos y el tipo de datos que se usaria en cada
momento en base a las necesidades del proyecto. Para sacar el máximo partido a las
funcionalidades que se han desarrollado.

\subsubsection{CII016}
\textit{"Conocimiento y aplicación de los principios, metodologías y ciclos de vida de la
ingeniería de software."}

\medskip
Se ha utilizando una metodología basada en el desarrollo incremental, puesto que el objetivo
inicial era que el cliente con cada entrega viera una nueva funcionalidad que probar. Con la
que poder recibir \textit{feedback} y realizar las correcciones o mejoras posibles.


\subsection{Ingeniería del Software (IS)}

\subsubsection{IS01}
\textit{"Capacidad para desarrollar, mantener y evaluar servicios y sistemas software que
satisfagan todos los requisitos del usuario y se comporten de forma fiable y
eficiente, sean asequibles de desarrollar y mantener y cumplan normas de
calidad, aplicando las teorías, principios, métodos y prácticas de la ingeniería del
software."}

\medskip
Durante el desarrollo del proyecto, se ha seguido una metodología y planificación adecuada
al cliente de manera que se han cumplido los objetivos marcados. Entregado software
funcional y de calidad, acordes con las necesidades del problema.

\subsubsection{IS02}
\textit{"Capacidad para valorar las necesidades del cliente y especificar los requisitos
software para satisfacer estas necesidades, reconciliando objetivos en
conflicto mediante la búsqueda de compromisos aceptables dentro de
las limitaciones derivadas del coste, del tiempo, de la existencia de sistemas ya
desarrollados y de las propias organizaciones."}

\medskip
En la primera fase del proyecto se le ha dado importancia a las necesidades del cliente
para conocer cuales son las funcionalidades más relevantes y con esto poder establecer
prioridades en las tareas y modificar algunas, las cuales, no encajaban con la visión
del cliente.

\medskip
Para ello se ha ido mostrando \textit{mockups} de la aplicación y diagramas. Además, durante
el desarrollo se ha cambiado ciertos objetivos, dado que inicialmente no se había visto
alguna problemática que ha surgido durante este.

\subsubsection{IS03}
\textit{"Capacidad de dar solución a problemas de integración en función de las estrategias,
estándares y tecnologías disponibles."}

\medskip
Dado que finalmente se ha decidido hacer la aplicación conectada con un proveedor de base de
datos externo, ha generado dificultades de integración que se han tenido que solucionar.

\subsubsection{IS04}
\textit{"Capacidad de identificar y analizar problemas y diseñar, desarrollar,
implementar, verificar y documentar soluciones software sobre la base de
un conocimiento adecuado de las teorías, modelos y técnicas actuales."}

\medskip
Para el desarrollo del proyecto se ha puesto en práctica distintas técnicas, principios y
modelos vistos durante el grado con el objetivo de obtener el mejor \textit{software} posible.
Como han sido los diagramas UML, diagramas de casos de uso, etc.