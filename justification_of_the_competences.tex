\section{Justificación de las competencias}
%  JUSTIFICAIÓN DE LAS COMPETENCIAS ESPECÍFICAS CUYBIERTAS: indicar, sólo para 
%  las competencias específicas relacionadas de forma más directa con el trabajo 
%  desarrollado, cómo se han cubierto con este TFT.
% COMPETENCIAS: CII08 – CII012 – CII016 – IS01 – IS02 – IS03 – IS04

\subsection{Comunes a la Ingeniería informática}
\subsubsection{CII08}
\textit{"Capacidad para analizar, diseñar, construir y mantener aplicaciones de forma
robusta, segura y eficiente, eligiendo el paradigma y los lenguajes de programación más
adecuados."}

\subsubsection{CII012}
\textit{"Conocimiento y aplicación de las características, funcionalidades y estructura de las
bases de datos, que permitan su adecuado uso, y el diseño y el análisis
e implementación de aplicaciones basadas en ellos."}

\subsubsection{CII016}
\textit{"Conocimiento y aplicación de los principios, metodologías y ciclos de vida de la
ingeniería de software."}

\subsubsection{IS01}
\textit{"Capacidad para desarrollar, mantener y evaluar servicios y sistemas software que
satisfagan todos los requisitos del usuario y se comporten de forma fiable y
eficiente, sean asequibles de desarrollar y mantener y cumplan normas de
calidad, aplicando las teorías, principios, métodos y prácticas de la ingeniería del
software."}

\subsubsection{IS02}
\textit{"Capacidad para valorar las necesidades del cliente y especificar los requisitos
software para satisfacer estas necesidades, reconciliando objetivos en
conflicto mediante la búsqueda de compromisos aceptables dentro de
las limitaciones derivadas del coste, del tiempo, de la existencia de sistemas ya
desarrollados y de las propias organizaciones."}

\subsubsection{IS03}
\textit{"Capacidad de dar solución a problemas de integración en función de las estrategias,
estándares y tecnologías disponibles."}

\subsubsection{IS04}
\textit{"Capacidad de identificar y analizar problemas y diseñar, desarrollar,
implementar, verificar y documentar soluciones software sobre la base de
un conocimiento adecuado de las teorías, modelos y técnicas actuales."}
