\section{Aportaciones al entorno}
%  APORTACIONES: justificar que es lo que este TFT aporta a nuestro entorno socio-economico,
%  tecnico o cientifico.
\subsection{Entorno socio-económico}
El desarrollo de este proyecto ofrecerá tanto a padres como doctores un medio de comunicación
y visualización de ejercicios rápido, sencillo y con el que evitar gasto de papel que conlleva
el sistema actual. A día de hoy a los padres se le dan hojas con las ilustraciones referentes
a los ejercicios que tienen que realizar.

\medskip
El gasto masivo de papel es un tema que está a la orden del día, puesto que de él deriva la
tala de árboles, la cual tiene varias consecuencias negativas:
\begin{enumerate}
    \item Cambios de clima debido a la falta de árboles para la retención de humedad, con lo que aumenta la posibilidad de sequías.
    \item Destrucción de ecosistemas y la perdida de biodiversidad que ello conlleva.
    \item Disminución del medio de transformación de dióxido de carbono en oxigeno, aumentando el efecto invernadero y aumentando las enfermedades respiratorias de la población.
\end{enumerate}

\bigskip
Además del reducir el impacto negativo en el medioambiente, reduce gastos en los centros
que hagan uso de la aplicación. Eliminando el gasto referente al papel y reduciendo el
tiempo que los doctores utilizan para solventar dudas del paciente debido a:
\begin{enumerate}
    \item Mayor rapidez en la comunicación doctor-paciente, pues con la solución propuesta esta comunicación es instantánea.
    \item Ahorro por parte de los pacientes en cuanto a desplazamiento en combustible y tiempo.
    \item Mayor facilidad para recordar el ejercicio, dado que se provee de material en forma de vídeo.
    \item Seguridad por parte del paciente, al tener material y medio de comunicación directo con su doctor que puede consultar en cualquier momento.
    \item Facilidad para el doctor a la hora de realizar seguimientos.
\end{enumerate}

\medskip
Este proyecto tiene consecuencias positivas tanto en el medio ambiente, como para los doctores
y familias con hijos prematuros.

\subsection{Personal}
A nivel personal, este proyecto resulta enriquecedor. Ha sido desarrollado desde cero
pasando por todas y cada una de las fases de un proyecto informático desde el análisis
hasta la implementación.

\medskip
Gracias a ello se ha podido reforzar los conocimientos adquiridos durante la carrera referentes
a cada una de estas fases de desarrollo, así como a las herramientas y tecnologías utilizadas. A
nivel de implementación la aplicación móvil a servido para conocer nuevas tecnologías como
\textit{Ionic}, \textit{TypeScript}, \textit{SCSS}, \textit{Firebase}.