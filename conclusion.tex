\section{Conclusiones y trabajos futuros}
\subsection{Conclusiones}

En términos generales los objetivos con los que se inició el proyecto han
sido cumplidos. Logrando una primera aproximación funcional para
la problemática planteada. Teniendo en cuenta el poco margen de tiempo del
que se dispone en este tipo de proyectos, se ha logrado una buena primera
versión que esperamos que pueda seguir evolucionando y mejorando con el tiempo.

\medskip
La participación y realización de este proyecto ha resultado una
experiencia enriquecedora y formativa. Se ha podido reforzar los
conocimientos adquiridos en la carrera tanto genéricos como específicos
de la Ingeniería del Software. Además de obtener nuevos referentes a
las tecnologías que se han decidido usar como Ionic, TypeScript, Firebase,
entre otras.

\medskip
Al tratarse de un proyecto que se ha iniciado desde cero hasta el final
del mismo ha sido una experiencia nueva, al no haber hecho antes
tal proceso al completo. Si es cierto que se conocían todas sus
fases por separado, pero el poder aplicarlas juntas ha resultado
una experiencia muy positiva.

\medskip
Además, haber buscado una solución a un problema
existente y haber sido capaz de aprender una tecnología nueva.
Me ha hecho ver que estos años de universidad
no solamente se han adquirido conocimientos técnicos. También ha
hecho que adquiera una forma de pensar con la que poder buscar soluciones a
problemas y una mentalidad abierta para aprender nuevas cosas
ya sean tecnologías o metodologías de forma más o menos rápida y
fácil.

\medskip
En términos generales este proyecto ha sido una experiencia
muy gratificante. A pesar de sus cambios en medio del desarrollo,
sus horas de aprendiendizaje de nuevas herramienta en las que parecía que
no se avanzaba con el proyecto. Las entrevistas
con el cliente y con todo lo ocurrido se puede decir que ha sido
una muy buena experiencia y se ha logrado obtener un resultado cercano al
esperado.

\subsection{Mejoras futuras}
A continuación se describirán mejoras a la funcionalidad
implementada y nuevas funcionalidades que se podrían incorporar, de
cara a completar este proyecto.
\begin{itemize}
    \item Enviar vídeos por el chat, funcionalidad que se tuvo que
    dejar por falta de tiempo y es necesaria para el producto
    mínimo viable.
    \item Permitir usuarios de tipo administrador, el cual pueda añadir
    ejercicios a la base de datos, modificar o eliminar los mismos.
    \item Añadir evaluación de la evolución del paciente, a día de hoy
    solo se puede ver cuando ha realizado el ejercicio.
    \item Enviar esta evaluación por el chat al paciente para que sea
    notificado.
    \item Permitir marcar ejercicio como hecho sin tener que entrar
    en la vista del ejercicio.
    \item Implementar un sistema de notificaciones si un ejercicio
    se ha marcado con una periodicidad diaria y hace más de 20 horas
    que no se ha indicado que se ha realizado.
    \item En el momento del registro permitir más localidades de
    nacimiento, no solamente municipios de Gran Canaria.
    \item Añadir un servicio de gestión de citas, donde el médico
    pueda indicar que tiene una cita con un paciente y este pueda
    ver en su teléfono cuando será la siguiente cita.
    \item Poder notificar tanto paciente como doctor que no podrá
    asistir a la cita.
    \item Notificación cuando la fecha de la cita prevista se acerca.
    \item Clasificar ejercicios por categorías, para ayudar a la búsqueda
    una vez se tenga una gran cantidad de ellos.
    \item Mejorar la estética de la interfaz de usuario, como puede ser,
    añadir un logotipo de la aplicación.
\end{itemize}